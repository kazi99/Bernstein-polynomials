\documentclass[a4paper, reqno]{amsart}
% \documentclass[a4paper,11pt]{article}

\usepackage[slovene]{babel}
\usepackage[T1]{fontenc}
\usepackage[utf8]{inputenc}
\usepackage{lmodern}

\usepackage{textcase}
\usepackage{amsmath}
\usepackage{amsfonts}
\usepackage{amsthm}
\usepackage{fancyhdr}
\usepackage[italicdiff]{physics}
\usepackage{url}
% \usepackage{hyperref}
% \usepackage{biblatex}
% \usepackage{mathabx}

\pagestyle{fancy}
\fancyhf{}
\rhead{\thepage}
\lhead{\textsc{Bernsteinovi polinomi in Weierstrassov izrek}}

\setlength{\parskip}{1em}

{
\theoremstyle{theorem}
\newtheorem{izrek}{Izrek}[section]
\newtheorem{lema}[izrek]{Lema}
\newtheorem{trditev}[izrek]{Trditev}
\newtheorem{posledica}[izrek]{Posledica}
}

{
\theoremstyle{definition}
\newtheorem{definicija}{Definicija}[section]
\newtheorem{primer}[definicija]{Primer}
\newtheorem*{zgled*}{Zgled}
\newtheorem*{opomba*}{Opomba}
}

%%%%%%%%%%%%%%%%%%%%%%%%%%%%%%%%%%%%%%%%%%%%%%%%%%%%%%%%%%%%%%%%%%%%%%%%

\begin{document}

\title{Bernsteinovi polinomi in Weierstrassov aproksimacijski izrek}
\author{Izak Jenko}
% \address{FMF}
% \date{marec 2020}

\maketitle
\thispagestyle{empty}

\begin{abstract}
	V tem članku bomo spoznali enega izmed prvih konstruktivnih dokazov 
	\emph{Weierstrassovega aproksimacijskega izreka}, ki pravi da lahko vsako zvezno 
	funkcijo na zaprtem intervalu poljubno dobro enakomerno aproksimiramo s polinomom. 
	Definirali bomo posebno vrsto polinomov t.~i. \emph{Bernsteinove polinome}, si ogledali
	njihove lastnosti in z njimi dokazali Weierstrassov izrek.
\end{abstract}

\newcommand{\RR}{\mathbb{R}}
\newcommand{\NN}{\mathbb{N}}
\newcommand{\CIR}{\mathcal{C}([0,1],\RR)}
\newcommand{\dinf}{d_{\infty}}

\section{Uvod}

\par
Preden se lotimo definicije Bernsteinovih polinomov in dokazovanja Weierstrassovega
izreka, povejmo nekaj besed o prostoru, ki ga bomo obranavali ter normi, ki bo
določila metriko na tem prostoru. Iz razlogov, ki jih bomo razjasnili pozneje, se bomo
osredotočili na prostor $\mathcal{C}([0,1],\mathbb{R})$, to je prostor vseh zveznih
funkcij $f: [0,1] \to \RR$, in supremum normo
$$\|f\|_{\infty} := \sup \{|f(x)|\mid x \in [0,1]\}$$
na njem. Vemo že, da ta norma porodi supremum metriko
$$\dinf (f, g) = 
% \|f - g\|_{\infty} =
\sup_{x \in [0,1]} |f(x) - g(x)| \text{,}$$
za katero velja, da na kompaktnem intervalu $[0,1]$ poljubno zaporedje zveznih funkcij 
$(f_n)_{n \in \NN} \subset \CIR$ enakomerno konvergira k neki zvezni funkciji 
$f$ iz prostora $\CIR$
natanko tedaj, ko to zaporedje $(f_n)_{n \in \NN}$ konvergira proti $f$ v metriki $\dinf$.
Ravno to dejstvo pa bomo uporabili pri dokazu Weierstrassovega izreka. 
\par
Posvetimo sedaj še nekaj pozornosti samemu pojmu aproksimacije ter se začasno usmerimo
na malce bolj splošne prostore. Naj bo $(X, d)$ poljuben metričen prostor in $A \subseteq X$
takšna podmnožica, za katero lahko povemo, da za poljuben $\epsilon > 0$ in vsak $x \in X$ 
obstaja v $A$ takšna točka $a \in A$, da je $$d(a, x) < \epsilon\text{.}$$ Lastnost množice
$A$ je tedaj ravno ta, da je mogoče vsako točko $x \in X$ poljubno dobro aproksimirati
z nekim elementom iz $A$. Pravzaprav takrat za to lastnost rečemo, da je množica $A$
povsod gosta v $X$. 

%%%%%%%%%%%%%%%%%%%%%%%%%%%%%%%%%%%%%%%%%%%%%%%%%%%%%%%%%%%%%%%%%%%%%%%%

\section{Bernsteinovi polinomi}

V tem poglavju bomo začeli z definicijo Bernsteinovih baznih polinomov in zanje
dokazali nekaj zanimivih lastnosti, pozneje pa bomo definirali Bernsteinov polinom
$n$-te stopnje pripadajoč zvezni neki funkciji $f \in \CIR$.

\begin{definicija}[\textbf{Bernsteinovi bazni polinomi}]
	\emph{Bernsteinov bazni polinom $n$-te stopnje} za poljuben $k \in \{0,1,\dots,n\}$ 
	je
	$$ b_{k, n}(x) = \binom nk x^k(1 - x)^{n - k}\text{.}$$
\end{definicija}
	
\begin{zgled*}
	Prvih nekaj Bernsteinovih baznih polinomov do tretje stopnje.
	\begin{align*}
		b_{0,0}(x) &= 1          & b_{0,1}(x) &= 1 - x       \\
							 &             & b_{1,1}(x) &= x  		     \\
							 &  					 & 					  &              \\
	  b_{0,2}(x) &= (1 - x)^2  & b_{0,3}(x) &= (1 - x)^3   \\
		b_{1,2}(x) &= 2x(1 - x)  & b_{1,3}(x) &= 3x(1 - x)^2 \\
		b_{2,2}(x) &= x^2        & b_{2,3}(x) &= 3x^2(1 - x) \\
							 &             & b_{3,3}(x) &= x^3   
	\end{align*}
\end{zgled*}

\begin{trditev}
	Bernsteinovi bazni polinomi $n$-te stopnje $\{b_{k,n}\}_{0 \leq k \leq n}$
	tvorijo bazo $(n+1)$-dimenzionalnega realnega vetorskega prostora polinomov 
	največ $n$-te stopnje $\RR_n[X]$ za poljubno naravno število $n \in \NN$.
\end{trditev}

\begin{proof}
	Vemo že, da je $\{1, x, x^2, \dots, x^n\}$ baza vektorskega prostora $\RR_n[X]$.
	Vidimo tudi, da je moč množice $\{b_{i,n}\}_{0 \leq i \leq n}$ enaka $n + 1$,
	torej bo dovolj če pokažemo, da lahko vsako potenco $x^k$ za $0 \leq k \leq n$
	izrazimo kot linearno kombinacjo Bernsteinovih baznih polinomov. Za poljuben 
	$k \in \{0, \dots, n\}$ velja namreč

	$$ x^k = \sum_{i = k}^{n} \frac{\binom ik}{\binom nk} b_{i, n}(x)\text{.}$$

	Preverimo to enačbo z računom.
	\begin{align*}
	x^k &= x^k(x + (1 - x))^{n - k} \\
	&= x^k \sum_{i = 0}^{n - k} \binom{n - k}{i} x^i(1 - x)^{n - k - i} \\
	&= \sum_{i = 0}^{n - k} \binom{n - k}{i} x^{k + i}(1 - x)^{n - k - i} \\
	&= \sum_{i = k}^{n} \binom{n - k}{i - k} x^i(1 - x)^{n - i} \text{,}\\
\intertext{
	kjer bomo uporabili zvezo $ \binom{n - k}{i - k} = \frac{(n - k)!}{(i - k)!(n - i)!} =
	\frac{i!}{k!(i - k)!} \frac{k! (n - k)!}{n!} \frac{n!}{i!(n - i)!} =
	\frac{\binom ik}{\binom nk} \binom ni $
}
	&= \sum_{i = k}^{n} \frac{\binom ik}{\binom nk} \binom ni x^i(1 - x)^{n - i} \\
	&= \sum_{i = k}^{n} \frac{\binom ik}{\binom nk} b_{i, n}(x)\text{.}
	\end{align*}
\end{proof}

\begin{trditev}
	\label{razclenitev enote}
	Bernsteinovi bazni polinomi stopnje $n$ tvorijo razčlenitev enote.

	$$ \sum_{k=0}^{n}b_{k, n}(x) \equiv 1$$

\end{trditev}

\begin{proof}
	Trditev enostavno preverimo z uporabo binomskega izreka in definicije 
	Bernsteinovih baznih polinomov.

	$$ \sum_{k=0}^{n}b_{k, n}(x) = 
	\sum_{k=0}^{n} \binom nk x^k(1 - x)^{n - k} = 
	(x + (1 - x))^n = 
	1$$
\end{proof}

Sedaj bomo definirali Bernsteinove polinome pripadajoče neki zvezni funkciji, 
ki jih bomo uporabli pri dokazu Weierstrassovega
izreka, poleg tega pa bomo dokazali še nekaj lem, ki se bodo pravtako izkazale za koristne
v dokazu glavnega izreka. 

\begin{definicija}
	\emph{Bernsteinov $n$-ti polinom} poljubne zvezne funkcije $f \in \CIR$ je
	$$ B_n(x, f) := \sum_{k = 0}^n f\left(\frac kn\right)b_{k,n}(x)\text{.}$$

\end{definicija}

\begin{lema}
	\label{monotonost}
	Naj bosta $f, g \in \CIR$ poljubni zvezni funkciji za kateri velja $f \leq g$, potem za
	njuna Bernsteinova polinoma poljubne stopnje $n$ velja

	$$ B_n(\text{- },f) \leq B_n(\text{- },g)\text{.}$$ 
\end{lema}

\begin{proof}
	Ker za poljuben $x \in [0,1]$ velja $f(x) \leq g(x)$, bo tudi za vsak $k \in \{0,1,\dots, n\}$
	veljalo
	$$ f\left(\frac kn\right) \leq g\left(\frac kn\right)\text{.}$$
	Uporabimo to v računu

	$$ B_n(x, f) = \sum_{k = 0}^n f\left(\frac kn\right)b_{k,n}(x) \leq 
	\sum_{k = 0}^n g\left(\frac kn\right)b_{k,n}(x) = B_n(x, g)\text{,}$$

	ki velja za vsak $x \in [0,1]$.
\end{proof}

\begin{lema}
	\label{linearnost}
	Naj bo $f \in \CIR$ poljubna zvezna funkcija in $\alpha, \beta \in \RR$ poljubni konstanti.
	Tedaj za vsak $x \in [0,1]$ velja 

	$$ B_n(x, \alpha \cdot f + \beta) = \alpha B_n(x, f) + \beta\text{.}$$
\end{lema}

\begin{proof}
	Enakost preprosto preverimo z računom
	\begin{multline*}
	B_n(x, \alpha \cdot f + \beta) = 
	\sum_{k = 0}^n (\alpha \cdot f + \beta)\left( k/n \right)b_{k,n}(x) =\\
	= \sum_{k = 0}^n (\alpha f\left(\frac kn \right) + \beta)b_{k,n}(x) =\\
	= \alpha \sum_{k = 0}^n f\left(\frac kn\right)b_{k,n}(x) + \beta \sum_{k = 0}^n b_{k,n}(x) =
	\alpha B_n(x, f) + \beta\text{,}
\end{multline*}
kar velja za vse $x \in [0,1]$, pri čemer pa smo uporabili trditev \ref{razclenitev enote}.
\end{proof}

\begin{primer}
	\label{B polinom primer}
	\cite{Kadison}
	Za poljuben $n \in \NN$ izračunajmo $B_n(x, t \mapsto t)$ in $B_n(x, t \mapsto t^2)$. 
	Po binomskem izreku vemo
$$
\pdv{p}\left( \sum_{k=0}^n \binom nk p^k q^{n-k}\right) = \pdv{p}\left((p + q)^n\right) =
n (p + q)^{n - 1}\text{,}
$$
torej, če obe strani pomnožimo s $\frac pn$, dobimo enačbo

\begin{equation}
	\sum_{k=0}^n \binom nk \frac kn p^k q^{n-k} = (p + q)^{n - 1}p\text{.}
	\label{eq:prviodvod}
\end{equation}

Če v to enačbo vstavimo $p = x$ in $q = 1 - x$ dobimo
$$
x = \sum_{k=0}^n \binom nk \frac kn x^k (1 - x)^{n-k} = 
\sum_{k = 0}^n \frac kn b_{k,n}(x) =
B_n(x, t \mapsto t)\text{.}
$$

Za izračun $B_n(x, t \mapsto t^2)$ ponovno parcialno odvajajmo 
enačbo~\eqref{eq:prviodvod} po spremenljivki $p$ in nato obe strani
pomnožimo s $\frac pn$. Tedaj dobimo
$$
\sum_{k=0}^n \binom nk \frac {k^2}{n^2} p^k q^{n-k} = 
\frac {(n - 1)(p + q)^{n - 2}}{n}p^2 + \frac{(p + q)^{n - 1}}{n}p\text{.}
$$
Podobno kot prej tudi v to enačbo vstavimo $p = x$ in $q = 1 - x$, zato
$$
\frac{(n - 1)x^2}{n} + \frac xn = 
\sum_{k=0}^n \binom nk \frac {k^2}{n^2} x^k (1 - x)^{n-k} = 
\sum_{k = 0}^n \frac{k^2}{n^2} b_{k,n}(x) =
B_n(x, t \mapsto t^2)\text{.}
$$

Ti dve formuli smo izračunali predvsem zato, ker ju bomo v naslednjem poglavju uporabili
v dokazu Bernsteinovega izreka.
\end{primer}

%%%%%%%%%%%%%%%%%%%%%%%%%%%%%%%%%%%%%%%%%%%%%%%%%%%%%%%%%%%%%%%%%%%%%%%%

\section{Weierstrassov aproksimacijski izrek}

\par
Prispeli smo do glavnega dela tega članka. Tu bomo formulirali znani Weierstrassov
izrek o enakomerni aproksimaciji zvezne funkcije s polinomi na zaprtem intervalu,
in ga dokazali tako, da bomo najprej pokazali kako iz Bernsteinovega izreka o 
enakomerni konvergenci izbranega zaporedja Bernsteinovih polinomov sledi 
Weierstrassov izrek, ter nato dokazali še resničnost prvega. 

\begin{izrek}[\textbf{Weierstrassov aproksimacijski izrek}]
	\label{Weierstrass}
	Za poljubno zvezno funkcijo $f: [a, b] \to \RR$ in poljubno natančnost $\epsilon > 0$
	% pozitivno število ?
	obstaja polinom $p$ na $[a,b]$ za katerega velja

	$$ \dinf(f, p) = \sup_{x \in [a, b]} |f(x)-p(x)| < \epsilon\text{.}$$
	
\end{izrek}
	
Podajmo še Bernsteinov izrek.

\begin{izrek}[\textbf{Bernstein}]
	\label{Bernsteinov izrek}
	Za poljubno zvezno funkcijo $f:[a, b] \to \RR$ zaporedje polinomov 
	$(B_n(\text{- },f))_{n \in \NN}$ enakomerno konvergira
	proti $f$ na $[a, b]$ oziroma

	$$ \lim_{n\to\infty}B_n(x, f) = f(x)$$

	enakomerno na $[a,b]$.
\end{izrek}

\begin{opomba*}
	Pokažimo, da iz izreka \ref{Bernsteinov izrek} sledi izrek \ref{Weierstrass}.
	Izberimo si poljuben $\epsilon > 0$. Ker zaporedje polinomov 
	$(B_n(\text{- },f))_{n \in \NN}$ enakomerno konvergira na $[a,b]$, obstaja tak 
	$N \in \NN$, da za vse $x \in [a,b]$ in vse $n \geq N$ velja
	$$ |B_n(x,f) - f(x)| < \epsilon\text{.}$$
	Torej po definiciji metrike $\dinf$ za dobljeni $N$ velja
	\begin{equation*}
		\dinf(f, B_N(\text{- },f)) < \epsilon \text{,}
	\end{equation*}
	kjer je iskani polinom $p$ seveda enak $B_N(\text{- },f)$ in to torej pokaže 
	Weierstrassov aproksimacijski izrek.
\end{opomba*}

\begin{proof}[Dokaz \emph{izreka \ref{Bernsteinov izrek}}]

\par
Izrek bomo dokazali za primer, ko je zaprti interval $[a,b] = [0,1]$. Pa utemeljimo
zakaj smemo to storiti. Naj bo $g \in \mathcal{C}([a,b],\RR)$ poljubna zvezna funkcija. 
Definirajmo preslikavo $\phi: [0,1] \to [a,b]$, podano s predpisom
$$ \phi(t) = (b - a)t + a \text{, }$$
ki je očitno homeomorfizem. Tedaj je funkcija $f = g \circ \phi$ zvezna in definirana
na $[0,1]$. Torej je res dovolj izrek dokazati za poljubno zvezno funkcijo $f \in \CIR$. 

\par
Naj bo $f : [0,1] \to \RR$ zvezna funkcija. Dokazujemo $\lim_{n\to\infty}B_n(x, f) = f(x)$
enakomerno na $[0,1]$, zato izberimo poljuben pozitiven $\epsilon > 0$. Najprej
omenimo nekaj opazk, ki takoj sledijo iz danih predpostavk.

\par
Ker je $f$ zvezna na zaprtem intervalu, ki je kompakten, je $f$ omejena \cite[Izrek 30]{Ana1Skripta}.
Zato obstaja tak $M \in \RR$, vzamemo lahko kar $M = \|f\|_{\infty}$, da na $[0,1]$ velja
$$ |f| \leq M \text{.}$$
\par
Vidimo tudi, da je $f$ enakomerno zvezna, saj je zvezna na zaprtem intervalu $[0,1]$ 
\cite[Izrek 28]{Ana1Skripta}.
Tedaj za naš zgoraj izbrani $\epsilon > 0$ obstaja tak $\delta > 0$, da za vsaka 
$x, y \in [0,1]$ velja
\begin{equation}
	\label{eq:enak.zv.}
	|x - y| < \delta \implies |f(x) - f(y)| < \frac{\epsilon}{2}\text{.}
\end{equation}

\par
Izberimo si sedaj poljuben $\xi \in [0,1]$. Zaradi zveznosti $f$ tedaj za vsak
$x \in [0,1]$, za katerega velja $|x - \xi| < \delta$, velja tudi 
$|f(x) - f(\xi)| < \frac{\epsilon}{2}$. Sicer pa, če je $x$ tak, da velja
$|x - \xi| \geq \delta$, v tem primeru je seveda $|x - \xi|/\delta \geq 1$, dobimo
\begin{equation}
	\label{eq:druga moznost}
	|f(x) - f(\xi)| \leq 2M \leq 2M\left(\frac{|x - \xi|^2}{\delta^2}\right) <
	2M\left(\frac{x - \xi}{\delta}\right)^2 + \frac{\epsilon}{2}\text{.}
\end{equation}

Zdaj združimo neenakosti \eqref{eq:enak.zv.} in \eqref{eq:druga moznost},
in vidimo, da za vsak $x \in [0,1]$ velja

\begin{equation}
	\label{eq:f neenakost}
	|f(x) - f(\xi)| < 2M\left( \frac{x - \xi}{\delta}\right)^2 + \frac{\epsilon}{2}\text{.}
\end{equation}

V naslednji račun bomo vpeljali Bernsteinove polinome, ki pa bodo zaenkrat še
nedoločene stopnje $n$, kateri bomo pozneje podali dodatne omejitve. Po lemah 
\ref{linearnost} in \ref{monotonost}, ki govorita o linearnosi in monotonosti 
Bernsteinovih polinomov v drugem argumentu, dobimo
\begin{align*}
	|B_(x, f) - f(\xi)| &= |B_n(x, f - f(\xi))|\\
	&\leq B_n(x, |f - f(\xi)|)\text{,}\\
\intertext{
		saj je $f - f(\xi) \leq |f - f(\xi)|$. In ker velja \eqref{eq:f neenakost}, sledi
}
	&\leq B_n(x, t \mapsto 2M\left(\frac{t - \xi}{\delta} \right)^2 + \frac{\epsilon}{2})\\
	&= \frac{2M}{\delta^2}B_n(x, t \mapsto (t - \xi)^2) + \frac{\epsilon}{2}\text{,}
\end{align*}
kjer smo v prvem in zadnjem koraku uporabili lemo \ref{linearnost} in v drugem ter
tretjem lemo \ref{monotonost}. Iz primera \ref{B polinom primer} lahko izračunamo
\begin{align*}
	B_n(x, t \mapsto (t - \xi)^2) &= 
	B_n(x, t \mapsto t^2) - 2\xi B_n(x, t \mapsto t) + \xi^2B_n(x, 1) \\
	&= \frac{(n - 1)x^2}{n} + \frac xn - 2\xi x + \xi^2 \\
	%&= x^2 + \frac{1}{n}(x - x^2) - 2\xi x + \xi^2 \\
	&= (x - \xi)^2 + \frac{1}{n}(x - x^2)
\end{align*}
Če torej nadaljujemo z oceno in uporabimo zgornjo enakost dobimo
$$
|B_n(x, f) - f(\xi)| \leq \frac{2M}{\delta^2}(x - \xi)^2 + 
\frac{2M}{\delta^2}\frac{1}{n}(x - x^2) + \frac{\epsilon}{2}\text{,}
$$
ki velja za vsak $x \in [0,1]$. Če v to enačbo vstavimo $x = \xi$ se ocena spremeni v
$$
|B_n(\xi, f) - f(\xi)| \leq  \frac{2M}{n\delta^2}(\xi - \xi^2) + \frac{\epsilon}{2}\text{,}
$$
ki pravtako velja za vsak $\xi \in [0,1]$, saj smo na začetku izbrali poljubnega. Hitro
se prepričamo, da izraz $\xi - \xi^2$ na intervalu $[0,1]$ doseže maksimum v $\frac{1}{4}$
in tako dobimo oceno
$$
|B_n(\xi, f) - f(\xi)| \leq \frac{\epsilon}{2} + \frac{M}{2n\delta^2}\text{,}
$$ 
ki velja za vse $\xi \in [0,1]$.

\par
Na tem mestu pa vzemimo $N = \left \lceil{\frac{M}{2\epsilon\delta^2}}\right \rceil$, saj
takrat za vsak $n \geq N$ dobimo
$$
\|B_n(\text{- }, f) - f\|_{\infty} < \epsilon\text{,}
$$
kar dokaže enakomerno konvergenco zaporedja $(B_n(\text{- },f))_{n \in \NN}$ na $[0,1]$.
\end{proof}

\par
Nekaj let po Bernsteinovem konstruktivističnem dokazu Weierstrassovega
izreka je posplošitev izreka leta 1937 formuliral in dokazal
Marshall H. Stone. V izreku je poljuben zaprt interval zamenjal z abstraktnim
topološkim prostorom $X$ in algebro polinomov s poljubno podalgebro
topološke algebre zveznih realnih funkcij $\mathcal{C}(X, \RR) = \mathcal{C}(X)$
opremljene s
kompaktno-odprto topologijo ter nekatermi posebnimi lastnostmi.
Naslednji izrek je iz knjige \cite[Izrek 4.28]{MrcunTop}.

\renewcommand{\labelenumi}{(\roman{enumi})}

\begin{izrek}[\textbf{Stone--Weierstrassov izrek}]
	\label{Stone}
	Naj bo $X$ topološki prostor in naj bo $A$ podalgebra algebre zveznih
	realnih funkcij $\mathcal{C}(X)$ na $X$, za katero velja:
	\begin{enumerate}
		\item vse konstantne realne funkcije na $X$ so v podalgebri $A$,
		\item za poljubni točki $x, y \in X\text{, če je } x \neq y$, potem obstaja takšna
		funkcija $f$ iz podalgebre $A$, da je $f(x) \neq f(y)$ (tej
		lastnosti pravimo tudi, da $A$ loči točke prostora $X$). 
	\end{enumerate}
	Tedaj je podalgebra $A$ gosta v $\mathcal{C}(X)$ glede na kompaktno-odprto
	topologijo. 
\end{izrek}

\par
Z dokazom tega posplošenega izreka se v tem članku ne bomo ukvarjali, vseeno
pa bi pokometirali kam v ta izrek sodijo Bernsteinovi polinomi oziroma kako
bi z njimi in s tem izrekom kot posledico ali poseben primer dokazali 
Weierstrassov izrek. To bomo storili v zadnjem zgledu.

\begin{zgled*}
	Kot zanimivost lahko še za konec na hitro preverimo, da podalgebra Bernsteinovih
	polinomov zares zadošča predpostavkama (i) in (ii) iz Stoneovega izreka 
	\ref{Stone}. Realno podalgebro Bernsteinovih polinomov lahko zapišemo kot
	$$
	\mathcal{B} = \{B_n(\text{- }, f) \mid f \in \CIR, n \in \NN\}\text{.}
	$$
	Predpostavki (i) zadostimo, ker je $B_1(x, 1) \equiv 1$, predpostavki (ii) pa zato,
	ker imamo injektiven polinom $B_6(x, t \mapsto t) \equiv x$ in ta vedno loči točke.
\end{zgled*}

\par
Videli smo torej, da je mogoče in na kakšen način se podaja ekspliciten predpis polinoma, 
ki je enakomerni približek poljubne zvezne funkcije na zaprtem intervalu. Izkaže pa se, 
da ta metoda aproksimacije ni precej dobra. To pomeni, da je za dano natančnost potrebno
podati polinom razmeroma visoke stopnje, oziroma da dano zaporedje Bernsteinovih
polinomov dokaj počasi konvergira k iskani zvezni funkciji. Iz tega razloga se 
Bernsteinovi polinomi niso prav zares uveljavili na področju aproksimiranja zveznih 
funkcij, so pa v svojem času skupaj z dokazom Weierstrassovega izreka izstopili 
ravno zaradi svoje konstruktivistične narave.  

%%%%%%%%%%%%%%%%%%%%%%%%%%%%%%%%%%%%%%%%%%%%%%%%%%%%%%%%%%%%%%%%%%%%%%%%
\nocite{*}
\newpage
\bibliographystyle{plain}

\bibliography{literatura}

\end{document}