\documentclass[a4paper]{amsart}
% \documentclass[a4paper,11pt]{article}

\usepackage[slovene]{babel}
\usepackage[T1]{fontenc}
\usepackage[utf8]{inputenc}
\usepackage{lmodern}

\usepackage{textcase}
\usepackage{amsmath}
\usepackage{amsfonts}
\usepackage{amsthm}
\usepackage{fancyhdr}

\pagestyle{fancy}
\fancyhf{}
\rhead{\thepage}
\lhead{\textsc{Bernsteinovi polinomi in Weierstrassov izrek}}

\setlength{\parskip}{1em}

{
\theoremstyle{theorem}
\newtheorem{izrek}{Izrek}[section]
\newtheorem{lema}[izrek]{Lema}
\newtheorem{trditev}[izrek]{Trditev}
\newtheorem{posledica}[izrek]{Posledica}
}

{
\theoremstyle{definition}
\newtheorem{definicija}{Definicija}[section]
\newtheorem*{zgled*}{Zgled}
\newtheorem*{opomba*}{Opomba}
}

%%%%%%%%%%%%%%%%%%%%%%%%%%%%%%%%%%%%%%%%%%%%%%%%%%%%%%%%%%%%%%%%%%%%%%%%

\begin{document}

\title{Bernsteinovi polinomi in Weierstrassov aproksimacijski izrek}
\author{Izak Jenko}
% \address{FMF}
% \date{marec 2020}

\maketitle
\thispagestyle{empty}

\begin{abstract}
	V tem članku bomo spoznali enega izmed prvih konstruktivnih dokazov 
	\emph{Weierstrassovega aproksimacijskega izreka}, ki pravi da lahko vsako zvezno 
	funkcijo na zaprtem intervalu poljubno dobro enakomerno aproksimiramo s polinomom. 
	Definirali bomo posebno vrsto polinomov t.~i. \emph{Bernsteinove polinome}, si ogledali
	njihove lastnosti in z njimi dokazali Weierstrassov izrek.
\end{abstract}

\newcommand{\RR}{\mathbb{R}}
\newcommand{\NN}{\mathbb{N}}
\newcommand{\CIR}{\mathcal{C}([0,1],\RR)}
\newcommand{\dinf}{d_{\infty}}

\section{Uvod}

\par
Preden se lotimo definicije Bernsteinovih polinomov in dokazovanja Weierstrassovega
izreka, povejmo nekaj besed o prosotru, ki ga bomo obranavali ter normi, ki bo
določila metriko na tem prostoru. Iz razlogov, ki jih bomo razjasnili pozneje, se bomo
osredotočili na prostor $\mathcal{C}([0,1],\mathbb{R})$, to je prostor vseh zveznih
funkcij $f: [0,1] \to \RR$, in supremum normo
$$\|f\|_{\infty} := \sup \{|f(x)|\mid x \in [0,1]\}$$
na njem. Vemo že, da ta norma porodi supremum metriko
$$\dinf (f, g) = 
% \|f - g\|_{\infty} =
\sup_{x \in [0,1]} |f(x) - g(x)| \text{,}$$
za katero velja, da na kompaktnem intervalu $[0,1]$ poljubno zaporedje zveznih funkcij 
$(f_n)_{n \in \NN} \subset \CIR$ enakomerno konvergira k neki zvezni funkciji 
$f$ iz prostora $\CIR$
natanko tedaj, ko to zaporedje $(f_n)_{n \in \NN}$ konvergira proti $f$ v metriki $\dinf$.
Ravno to dejstvo pa bomo uporabili pri dokazu Weierstrassovega izreka. 
\par
Posvetimo sedaj še nekaj pozornosti samemu pojmu aproksimacije ter se začasno usmerimo
na malce bolj splošne prostore. Naj bo $(X, d)$ poljuben metričen prostor in $A \subseteq X$
takšna podmnožica, za katero lahko povemo, da za poljuben $\epsilon > 0$ in vsak $x \in X$ 
obstaja v $A$ takšna točka $a \in A$, da je $$d(a, x) < \epsilon\text{.}$$ Lastnost množice
$A$ je tedaj ravno ta, da je mogoče vsako točko $x \in X$ poljubno dobro aproksimirati
z nekim elementom iz $A$. Pravzaprav takrat za to lastnost rečemo, da je množica $A$
povsod gosta v $X$. 

% Sedaj pa si poglejmo primer konstrukcije posebnih polinomov, ki smo jih sicer že srečali
% pri Algebri 1, vendar pa bomo vildeli, da ne bodo služili kot dobra aproksimacija
% neke zvezne funkcije. Govorimo seveda o Lagrangevi interpolaciji.


%%%%%%%%%%%%%%%%%%%%%%%%%%%%%%%%%%%%%%%%%%%%%%%%%%%%%%%%%%%%%%%%%%%%%%%%
\section{Bernsteinovi polinomi}

V tem poglavju bomo začeli z definicijo Bernsteinovih baznih polinomov in zanje
dokazali nekaj zanimivih lastnosti, pozneje pa bomo definirali Bernsteinov polinom
$n$-te stopnje pripadajoč zvezni neki funkciji $f \in \CIR$.

\begin{definicija}[\textbf{Bernsteinovi bazni polinomi}]
	\emph{Bernsteinov bazni polinom $n$-te stopnje} za poljuben $k \in \{0,1,\dots,n\}$ 
	je
	$$ b_{k, n}(x) = \binom nk x^k(1 - x)^{n - k}\text{.}$$
\end{definicija}
	
\begin{zgled*}
	Prvih nekaj Bernsteinovih baznih polinomov do tretje stopnje.
	\begin{align*}
		b_{0,0}(x) &= 1          & b_{0,1}(x) &= 1 - x       \\
							 &             & b_{1,1}(x) &= x  		     \\
							 &  					 & 					  &              \\
	  b_{0,2}(x) &= (1 - x)^2  & b_{0,3}(x) &= (1 - x)^3   \\
		b_{1,2}(x) &= 2x(1 - x)  & b_{1,3}(x) &= 3x(1 - x)^2 \\
		b_{2,2}(x) &= x^2        & b_{2,3}(x) &= 3x^2(1 - x) \\
							 &             & b_{3,3}(x) &= x^3   
	\end{align*}
\end{zgled*}

\begin{trditev}
	Bernsteinovi bazni polinomi $n$-te stopnje $\{b_{k,n}\}_{0 \leq k \leq n}$
	tvorijo bazo $(n+1)$-dimenzionalnega realnega vetorskega prostora polinomov 
	največ $n$-te stopnje $\RR_n[X]$ za poljubno naravno število $n \in \NN$.
\end{trditev}

\begin{proof}
	Vemo že, da je $\{1, x, x^2, \dots, x^n\}$ baza vektorskega prostora $\RR_n[X]$.
	Vidimo tudi, da je moč množice $\{b_{i,n}\}_{0 \leq i \leq n}$ enaka $n + 1$,
	torej bo dovolj če pokažemo, da lahko vsako potenco $x^k$ za $0 \leq k \leq n$
	izrazimo kot linearno kombinacjo Bernsteinovih baznih polinomov. Za poljuben 
	$k \in \{0, \dots, n\}$ velja namreč

	$$ x^k = \sum_{i = k}^{n} \frac{\binom ik}{\binom nk} b_{i, n}(x)\text{.}$$

	Preverimo to enačbo z računom.
	\begin{align*}
	x^k &= x^k(x + (1 - x))^{n - k} \\
	&= x^k \sum_{i = 0}^{n - k} \binom{n - k}{i} x^i(1 - x)^{n - k - i} \\
	&= \sum_{i = 0}^{n - k} \binom{n - k}{i} x^{k + i}(1 - x)^{n - k - i} \\
	&= \sum_{i = k}^{n} \binom{n - k}{i - k} x^i(1 - x)^{n - i} \text{,}\\
\intertext{
	kjer bomo uporabili zvezo $ \binom{n - k}{i - k} = \frac{(n - k)!}{(i - k)!(n - i)!} =
	\frac{i!}{k!(i - k)!} \frac{k! (n - k)!}{n!} \frac{n!}{i!(n - i)!} =
	\frac{\binom ik}{\binom nk} \binom ni $
}
	&= \sum_{i = k}^{n} \frac{\binom ik}{\binom nk} \binom ni x^i(1 - x)^{n - i} \\
	&= \sum_{i = k}^{n} \frac{\binom ik}{\binom nk} b_{i, n}(x)\text{.}
	\end{align*}

% \footnotetext{ Velja zveza
% 	$ \frac{\binom ik}{\binom nk} \binom ni = 
% 	\frac{i!}{k!(i - k)!} \frac{k! (n - k)!}{n!} \frac{n!}{i!(n - i)!} = 
% 	\frac{(n - k)!}{(n - i)!(i - k)!} = \binom{n - k}{i - k}$
% 	}
% 	$ \binom{n - k}{i - k} = \frac{(n - k)!}{(n - i)!(i - k)!} =
% 	\frac{i!}{k!(i - k)!} \frac{k! (n - k)!}{n!} \frac{n!}{i!(n - i)!} =
% 	\frac{\binom ik}{\binom nk} \binom ni $
\end{proof}

\begin{trditev}
	\label{razclenitev enote}
	Bernsteinovi bazni polinomi stopnje $n$ tvorijo razčlenitev enote.

	$$ \sum_{k=0}^{n}b_{k, n}(x) \equiv 1$$

\end{trditev}

\begin{proof}
	Trditev enostavno preverimo z uporabo binomskega izreka in definicije 
	Bernsteinovih baznih polinomov.

	$$ \sum_{k=0}^{n}b_{k, n}(x) = 
	\sum_{k=0}^{n} \binom nk x^k(1 - x)^{n - k} = 
	(x + (1 - x))^n = 
	1$$

\end{proof}

% mogoče še kakšne irelevantne posledice če bo treba 

Sedaj bomo definirali Bernsteinove polinome pripadajoče neki zvezni funkciji, 
ki jih bomo uporabli pri dokazu Weierstrassovega
izrek, poleg tega pa bomo dokazali še nekaj lem, ki se bodo pravtako izkazale za koristne
v dokazu glavnega izreka. 

\begin{definicija}
	\emph{Bernsteinov $n$-ti polinom} poljubne zvezne funkcije $f \in \CIR$ je
	$$ B_n(x, f) := \sum_{k = 0}^n f\left(\frac kn\right)b_{k,n}(x)\text{.}$$

\end{definicija}

\begin{lema}
	Naj bosta $f, g \in \CIR$ poljubni zvezni funkciji za kateri velja $f \leq g$, potem za
	njuna Bernsteinova polinoma poljubne stopnje $n$ velja

	$$ B_n(\text{- },f) \leq B_n(\text{- },g)\text{.}$$ 
\end{lema}

\begin{proof}
	Ker za poljuben $x \in [0,1]$ velja $f(x) \leq g(x)$, bo tudi za vsak $k \in \{0,1,\dots, n\}$
	veljalo
	$$ f\left(\frac kn\right) \leq g\left(\frac kn\right)\text{.}$$
	Uporabimo to v računu

	$$ B_n(x, f) = \sum_{k = 0}^n f\left(\frac kn\right)b_{k,n}(x) \leq 
	\sum_{k = 0}^n g\left(\frac kn\right)b_{k,n}(x) = B_n(x, g)\text{,}$$

	ki velja za vsak $x \in [0,1]$.
\end{proof}

\begin{lema}
	Naj bo $f \in \CIR$ poljubna zvezna funkcija in $\alpha, \beta \in \RR$ poljubni konstanti.
	Tedaj za vsak $x \in [0,1]$ velja 

	$$ B_n(x, \alpha \cdot f + \beta) = \alpha B_n(x, f) + \beta\text{.}$$
\end{lema}

\begin{proof}
	Enakost preprosto preverimo z računom
	\begin{multline*}
	B_n(x, \alpha \cdot f + \beta) = 
	\sum_{k = 0}^n (\alpha \cdot f + \beta)\left(\frac kn\right)b_{k,n}(x) =\\
	\alpha \sum_{k = 0}^n f\left(\frac kn\right)b_{k,n}(x) + \beta \sum_{k = 0}^n b_{k,n}(x) =
	\alpha B_n(x, f) + \beta\text{,}
\end{multline*}

kar velja za vse $x \in [0,1]$, pri čemer pa smo uporabili trditev \ref{razclenitev enote}.
\end{proof}


%%%%%%%%%%%%%%%%%%%%%%%%%%%%%%%%%%%%%%%%%%%%%%%%%%%%%%%%%%%%%%%%%%%%%%%%
\section{Weierstrassov aproksimacijski izrek}

\begin{izrek}
	Za poljubno zvezno funkcijo $f: [a, b] \to \RR$ in pojubno natančnost $\epsilon > 0$
	obstaja polinom $p$ na $[a,b]$ za katerega velja:

	$$ \dinf(f, p) = \sup_{x \in [a, b]} |f(x)-p(x)| < \epsilon\text{.}$$

	Ekvivalentna formulacija, z uporabo Bernsteinovih polinomov:

	Za poljubno zvezno funkcijo $f:[a, b] \to \RR$ zaporedje polinomov 
	$(B_n(\text{- },f))_{n \in \NN}$ enakomerno konvergira
	proti $f$ na $[a, b]$ ali:

	$$ \lim_{n\to\infty}B_n(x, f) = f(x)$$
	
	enakomerno na $[a,b]$.
\end{izrek}

\par
Nekaj let po Bernsteinovem konstruktivističnem dokazu Weierstrassovega
izreka je posplošitev izreka leta 1937 formuliral in dokazal
Marshall H. Stone. V izreku je poljuben zaprt interval zamenjal z abstraktnim
topološkim prostorom $X$ in algebro polinomov s poljubno podalgebro
topološke algebre zveznih realnih funkcij $\mathcal{C}(X, \RR) = \mathcal{C}(X)$
opremljene s
kompaktno odprto topologijo ter nekatermi posebnimi lastnostmi.

\renewcommand{\labelenumi}{(\roman{enumi})}

\begin{izrek}[\textbf{Stone--Weierstrassov izrek}] %mrcun
	Naj bo $X$ topološki prostor in naj bo $A$ podalgebra algebre zveznih
	realnih funkcij $\mathcal{C}(X)$ na $X$, za katero velja:
	\begin{enumerate}
		\item vse konstantne realne funkcije na $X$ so v podalgebri $A$,
		\item za poljubni točki $x, y \in X\text{,} x \neq y$, obstaja takšna
					funkcija $f$ iz podalgebre $A$, da je $f(x) \neq f(y)$ (tej
					lastnosti pravimo tudi, da $A$ loči točke prostora $X$). 
	\end{enumerate}
	Tedaj je podalgebra $A$ gosta v $\mathcal{C}(X)$ glede na kompaktno-odprto
	topologijo.
\end{izrek}

\par
Z dokazom tega posplošenega izreka se v tem članku ne bomo ukvarjali, vseeno
pa bi pokometirali kam v ta izrek sodijo Bernsteinovi polinomi oziroma kako
bi s tem izrekom kot posledico ali poseben primer dokazali Weierstrassov izrek. 








%%%%%%%%%%%%%%%%%%%%%%%%%%%%%%%%%%%%%%%%%%%%%%%%%%%%%%%%%%%%%%%%%%%%%%%%
\newpage
\section{Testi}

\begin{definicija}[Brokoli]
	To je definicija
\end{definicija}

\begin{lema}
	To je lema
\end{lema}

\begin{izrek}
	Drugi izrek
\end{izrek}

\begin{izrek}[\textbf{Weierstrassov izrek}]
	Prvi izrek
\end{izrek}

\begin{lema}
	Druga lema
\end{lema}

\begin{izrek}
	Tretji izrek
\end{izrek}

\begin{zgled*}
	$1 + 1 = 0$, če $\text{char}(K) = 2$
\end{zgled*}

\begin{posledica}
	p
\end{posledica}

\begin{zgled*}
	Prvih nekaj Bernsteinovih baznih polinomov do tretje stopnje.
		\begin{align*}
			b_{0,0}(x) &= 1\text{,}          &             &               
			&             &               &             &  \\
			b_{0,1}(x) &= 1 - x\text{,}      &  b_{1,1}(x) &= x\text{,}            
			&             &               &             &  \\
			b_{0,2}(x) &= (1 - x)^2\text{,}  &  b_{1,2}(x) &= 2x(1 - x)\text{,}    
			&  b_{2,2}(x) &= x^2\text{,}          &             &  \\ 
			b_{0,3}(x) &= (1 - x)^3\text{,}  &  b_{1,3}(x) &= 3x(1 - x)^2\text{,}  
			&  b_{2,3}(x) &= 3x^2(1 - x)\text{,}  &  b_{3,3}(x) &= (1 - x)^3\text{.}
		\end{align*}
	\end{zgled*}

$ a \cdot b$

%%%%%%%%%%%%%%%%%%%%%%%%%%%%%%%%%%%%%%%%%%%%%%%%%%%%%%%%%%%%%%%%%%%%%%%%
\newpage
\texttt{j}j

\end{document}