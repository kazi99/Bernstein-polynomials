\documentclass[a4paper]{amsart}
% \documentclass[a4paper,11pt]{article}

\usepackage[slovene]{babel}
\usepackage[T1]{fontenc}
\usepackage[utf8]{inputenc}
\usepackage{lmodern}

\usepackage{textcase}
\usepackage{amsmath}
\usepackage{amsfonts}
\usepackage{amsthm}
\usepackage{fancyhdr}

\pagestyle{fancy}
\fancyhf{}
\rhead{\thepage}
\lhead{\textsc{Bernsteinovi polinomi in Weierstrassov izrek}}


{
\theoremstyle{theorem}
\newtheorem{izrek}{Izrek}[section]
\newtheorem{lema}[izrek]{Lema}
\newtheorem{posledica}[izrek]{Posledica}
}

{
\theoremstyle{definition}
\newtheorem{definicija}{Definicija}[section]
\newtheorem*{zgled*}{Zgled}
}

%%%%%%%%%%%%%%%%%%%%%%%%%%%%%%%%%%%%%%%%%%%%%%%%%%%%%%%%%%%%%%%%%%%%%%%%

\begin{document}

\title{Bernsteinovi polinomi in Weierstrassov aproksimacijski izrek}
\author{Izak Jenko}
% \address{FMF}
% \date{marec 2020}

\maketitle
\thispagestyle{empty}

\begin{abstract}
	V tem članku bomo spoznali enega izmed prvih konstruktivnih dokazov 
	\emph{Weierstrassovega aproksimacijskega izreka}, ki pravi da lahko vsako zvezno 
	funkcijo na zaprtem intervalu poljubno dobro enakomerno aproksimiramo s polinomom. 
	Definirali bomo posebno vrsto polinomov t.~i. \emph{Bernsteinove polinome}, si ogledali
	njihove lastnosti in z njimi dokazali Weierstrassov izrek.
\end{abstract}

\newcommand{\RR}{\mathbb{R}}
\newcommand{\NN}{\mathbb{N}}
\newcommand{\CIR}{\mathcal{C}([0,1],\RR)}

\section{Uvod}

Preden se lotimo definicije Bernsteinovih polinomov in dokazovanja Weierstrassovega
izreka, povejmo nekaj besed o prosotru, ki ga bomo obranavali ter normi, ki bo
določila metriko na tem prostoru. Iz razlogov, ki jih bomo razjasnili pozneje, se bomo
osredotočili na prostor $\mathcal{C}([0,1],\mathbb{R})$ in supremum normo
$\|f\|_{\infty} =$ $\sup \{|f(x)|\mid x \in [0,1]\}$ na njem. Vemo že, da ta norma porodi 
supremum metriko $d_{\infty}(f, g) = sup \{|f(x) - g(x)| \mid x \in [0,1]\}$, 
za katero velja, da na kompaktnem intervalu $[0,1]$ zaporedje zveznih funkcij 
$(f_n)_{n \in \NN} \subset \CIR$ enakomerno konvergira k neki zvezni funkciji $f \in \CIR$
natanko tedaj ko to zaporedje $(f_n)_{n \in \NN}$ konvergira proti $f$ v metriki $d_\infty$.
Natanko to dejstvo bomo uporabili pri dokazu Weierstrassovega izreka.








%%%%%%%%%%%%%%%%%%%%%%%%%%%%%%%%%%%%%%%%%%%%%%%%%%%%%%%%%%%%%%%%%%%%%%%%
\newpage

\begin{definicija}[Brokoli]
	To je definicija
\end{definicija}

\begin{lema}
	To je lema
\end{lema}

\begin{izrek}
	Drugi izrek
\end{izrek}

\begin{izrek}[\textbf{Weierstrassov izrek}]
	Prvi izrek
\end{izrek}

\begin{lema}
	Druga lema
\end{lema}

\begin{izrek}
	Tretji izrek
\end{izrek}

\begin{zgled*}
	$1 + 1 = 0$, če $\text{char}(K) = 2$
\end{zgled*}

\begin{posledica}
	p
\end{posledica}

%%%%%%%%%%%%%%%%%%%%%%%%%%%%%%%%%%%%%%%%%%%%%%%%%%%%%%%%%%%%%%%%%%%%%%%%
\newpage
j

\end{document}